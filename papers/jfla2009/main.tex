\documentclass[twoside]{article}
\usepackage[latin1]{inputenc}
\usepackage[T1]{fontenc}
\usepackage{actes,alltt,url}
\usepackage[french]{babel}

% macros
\newcommand{\mlpost}{\textsc{Mlpost}}

\title{Faire bonne figure avec \mlpost}

\author{R. Bardou$^1$
        \& J. Kanig$^1$
        \& J.-C. Filli�tre$^1$
        \& S. Lescuyer$^1$}

\titlehead{Faire bonne figure avec \mlpost}%  a droite (page impaire)

\authorhead{Bardou \& Kanig \& Filli�tre \& Lescuyer}% a gauche (page paire)

\affiliation{\begin{tabular}{rr} 
\\ 1:  ProVal / INRIA Saclay -- �le-de-France
\\     91893 Orsay Cedex, France
\\     LRI / CNRS -- Universit� Paris Sud
\\     91405 Orsay Cedex, France
\\     {\tt \{bardou,kanig,filliatr,lescuyer\}@lri.fr} 
\end{tabular}}

\begin{document}
\setcounter{page}{1}
\maketitle

\begin{abstract}
  
\end{abstract}

\section{Introduction}

motivation (dessiner en programmant, inclusion latex)
\subsection{Les solutions existantes}

Il existent de nombreuses solutions pour des dessin techniques. Tout d'abord,
les logiciels � base d'interface graphique, notamment {\em dia} et {\em xfig}.
Ces deux programmes permettent la cr�ation � la souris de dessins plus ou
moins complexes. Ils ont l'avantage d'�tre assez intuitifs, mais ils partagent
aussi l'absence d'inclusion de \LaTeX dans un dessin. Avec xfig, on peut
n�anmoins mettre du code \LaTeX dans un document, mais sa taille r�elle ne
pourra �tre prise en compte. Dia n'offre m�me pas cette possibilit�, il faut
�diter les fichiers de sortie, si on veut vraiment mettre du code \LaTeX.

Les solutions � base de langages de programmation int�grent toutes une bon
support pour du texte en \LaTeX. Nous en citons les plus connus.

\begin{itemize}
  \item {\em pstricks} est un ensemble de macros \LaTeX. M�me si c'est
    relativement puissant, son utilisation est assez lourde et non-intuitive.
  \item tikz
  \item metapost
  \item mlpictex
  \item functional metapost
\end{itemize}<++>


solutions existantes:

\cite{metapost}

\section{Exemples}

des plus simples aux plus compliqu�s

\section{Architecture Logicielle}

API
compilation, hashconsing
modules r�cursifs - interface de MlPost
exemples compliqu�s expliqu�s
choix de la persistence

\begin{ocaml}
  type t = int -> int
\end{ocaml}


\paragraph{Remerciements.} Les auteurs tiennent � remercier Florence
Plateau, Yannick Moy et Claude March� pour leur contribution � \mlpost.

% La bibliographie
% N'oubliez pas de l'inclure lors de votre soumission.
\bibliographystyle{plain}
\bibliography{./biblio}

\vfill

\pagebreak
\thispagestyle{colloquetitle}
\cleardoublepage
\end{document}

% Local Variables:
% coding: iso-latin-1
% ispell-local-dictionary: "francais"
% End:
