\documentclass[twoside]{article}
\usepackage[latin1]{inputenc}
\usepackage[T1]{fontenc}
\usepackage{actes,alltt,url,graphicx}
\usepackage[french]{babel}

%; whizzy -pdf .

% macros
\newcommand{\mlpost}{\textsc{Mlpost}}
\newcommand{\metapost}{\textsc{MetaPost}}
\newcommand{\tikz}{\textsc{Tikz}}
\newcommand{\pstricks}{\textsc{Pstricks}}

\title{Faire bonne figure avec \mlpost}

\author{R. Bardou$^1$
        \& J. Kanig$^1$
        \& J.-C. Filli�tre$^1$
        \& S. Lescuyer$^1$}

\titlehead{Faire bonne figure avec \mlpost}%  a droite (page impaire)

\authorhead{Bardou \& Kanig \& Filli�tre \& Lescuyer}% a gauche (page paire)

\affiliation{\begin{tabular}{rr} 
\\ 1:  ProVal / INRIA Saclay -- �le-de-France
\\     91893 Orsay Cedex, France
\\     LRI / CNRS -- Universit� Paris Sud
\\     91405 Orsay Cedex, France
\\     {\tt \{bardou,kanig,filliatr,lescuyer\}@lri.fr} 
\end{tabular}}

\begin{document}
\setcounter{page}{1}
\maketitle

\begin{abstract}
  
\end{abstract}

\section{Introduction}

motivation (dessiner en programmant, inclusion latex)
\subsection{Les solutions existantes}

Il existent de nombreuses solutions pour des dessins techniques. Tout d'abord,
les logiciels � base d'interface graphique, notamment {\em dia} et {\em xfig}.
Ces deux programmes permettent la cr�ation � la souris de dessins plus ou
moins complexes. Ils ont l'avantage d'�tre assez intuitifs, mais ils partagent
aussi l'absence d'inclusion de \LaTeX dans un dessin. Avec xfig, on peut
n�anmoins mettre du code \LaTeX dans un document, mais sa taille r�elle ne
pourra �tre prise en compte. Dia n'offre m�me pas cette possibilit�, il faut
�diter les fichiers de sortie, si on veut vraiment mettre du code \LaTeX.

Les solutions � base de langages de programmation int�grent toutes une bon
support pour du texte en \LaTeX. Nous en citons les plus connus.

\begin{itemize}
  \item {\em pstricks} est un ensemble de macros \LaTeX. M�me si c'est
    relativement puissant, son utilisation est assez lourde et non-intuitive.
  \item tikz
  \item metapost
  \item mlpictex
  \item functional metapost
\end{itemize}<++>


solutions existantes:

\cite{metapost}

\section{Principes et exemples}

choix de la persistance

placement relatif plut�t qu'absolu

% automate -- RB
\begin{minipage}{0.2\linewidth}
  \includegraphics{automate.mps}
\end{minipage}
\begin{minipage}{0.2\linewidth}
\small\begin{ocaml}
open Num
let automate =
  let etat = Box.tex ~style: Circle in
  let final = Box.box ~style: Circle ~dx:zero ~dy:zero in
  let etats = Box.vbox ~padding: (cm 0.8) [
    Box.hbox ~padding: (cm 1.4) [ etat "$\\alpha$"; etat "$\\beta$" ];
    final (etat "$\\gamma$");
  ]in
  let alpha = nth 0 (nth 0 etats) in
  let beta = nth 1 (nth 0 etats) in
  let gamma = nth 1 etats in [
    Box.draw etats;
    Arrow.draw ~tex: "a" ~pos: `Lowleft (cpath alpha gamma);
    Arrow.draw ~tex: "b" ~pos: `Lowright (cpath gamma beta);
    Arrow.draw ~tex: "c" ~pos: `Top
      (cpath ~outd: (vec (dir 25.)) ~ind: (vec (dir 335.)) alpha beta);
    Arrow.draw ~tex: "d" ~pos: `Bot
      (cpath ~outd: (vec (dir 205.)) ~ind: (vec (dir 155.)) beta alpha);
    let w = Box.west alpha in
    Arrow.draw (Path.pathp [ Point.shift w (Point.pt (cm (-0.3), zero)); w ]);
  ]
\end{ocaml}
\end{minipage}

% blocs m�moire (listes, etc.) -- JCF

% diagramme simple -- JK

% diagramme UML -- JK
\begin{minipage}{0.3\linewidth}
  \includegraphics{uml.mps}
\end{minipage}
\begin{minipage}{0.3\linewidth}
\small\begin{ocaml}
let uml = 
  let classblock name attr_list method_list = 
    let tex = Box.tex ~stroke:None in
    let vbox = Box.vbox ~pos:`Left in
      Box.vblock ~pos:`Left ~name
      [ tex ("{\\bf " ^ name ^ "}");
        vbox (List.map tex attr_list); vbox (List.map tex method_list) ]
  in
  let a = classblock "BankAccount" 
            [ "balance : Dollars = $0$"] 
            ["deposit (amount : Dollars)"; "withdraw (amount : Dollars)" ] 
  in
  let b = classblock "Client" ["name : String"; "address : String" ] [] in
  let diag = Box.vbox ~padding:(Num.bp 50.) [a;b] in
  [ Box.draw diag; 
    box_label_arrow ~pos:`Left (Picture.tex "owns") 
      (get "Client" diag) (get "BankAccount" diag) ]
\end{ocaml}
\end{minipage}

% arbre -- JCF

% plot -- SL

\begin{minipage}{0.3\linewidth}
  \includegraphics{graph_sqrt.mps}
\end{minipage}
\begin{minipage}{0.3\linewidth}
\small\begin{ocaml}
let graph_sqrt =
  let u = cm 1. in
  let sk = Plot.mk_skeleton 4 2 u u in
  let label = Picture.tex "$\\sqrt x$", `Top, 3 in
  let graph = Plot.draw_func ~label (fun x -> sqrt (float x)) sk in
    [graph; Plot.draw_simple_axes "$x$" "$y$" sk]
\end{ocaml}
\end{minipage}

% exemples utilisant des calculs en Caml

% bresenham -- JCF

% exemples Florence et Yannick (sans le code)

\section{Architecture logicielle}

% sch�ma de l'architecture


\subsection{Types primitifs de \metapost}

API
compilation, hashconsing
modules r�cursifs - interface de MlPost

\subsection{Bo�tes}

- principe r�cursif : bo�te = picture ou ensemble de bo�tes + contour
+ fond

- bo�tes primitives
- r�union, alignements

\subsubsection{Application : arbres}

\subsection{Fl�ches} %%% RB

\section{Conclusion}

% pas fait / � faire

r�solution d'�quations

perspectives infinies d'extensions : 3D, animations, ...
plus facile de le faire en Caml qu'en \metapost\ ou en \LaTeX\ (\tikz,
\pstricks) 

plein de sortes de fl�ches

\paragraph{Remerciements.} Les auteurs tiennent � remercier Florence
Plateau, Yannick Moy et Claude March� pour leur contribution � \mlpost.

% La bibliographie
% N'oubliez pas de l'inclure lors de votre soumission.
\bibliographystyle{plain}
\bibliography{./biblio}

\vfill

\pagebreak
\thispagestyle{colloquetitle}
\cleardoublepage
\end{document}

% Local Variables:
% coding: iso-latin-1
% ispell-local-dictionary: "francais"
% End:
