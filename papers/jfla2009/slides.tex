\documentclass{beamer}

\usepackage[latin1]{inputenc}
\usepackage[T1]{fontenc}
\usepackage[french]{babel}
\usepackage{mflogo} 

\newcommand{\mlpost}{\textsc{Mlpost}}
\newcommand{\metapost}{\MP}

\begin{document}

\begin{frame}
  TITRE
\end{frame}

\begin{frame}\frametitle{}
 motivations

  - FIGURES + LATEX

  - pas de nouveau langage � apprendre

\end{frame}

\begin{frame}\frametitle{}
  quelles solutions
\end{frame}

\begin{frame}\frametitle{}
  notre solution
\end{frame}

\begin{frame}\frametitle{}
\begin{center}
\includegraphics[width=\textwidth]{stages.mps}
\end{center}
\end{frame}

\begin{frame}\frametitle{}
  d�mo
\end{frame}

\begin{frame}\frametitle{}

CHOIX DE CONCEPTION / API

- persistance

- arguments optionnels / �tiquet�s 

- variants polymorphes

- minimalit� de l'API

\end{frame}

\begin{frame}\frametitle{}
  \begin{center}
    \includegraphics{architecture.mps}
  \end{center}
\end{frame}

\begin{frame}\frametitle{}
  probl�mes techniques et solutions

  - d�pendances circulaires

  - limitations de \metapost\ $Rightarrow$ compilation

  - taille 
\end{frame}


\begin{frame}\frametitle{}
utilisations de \mlpost

- Florence, Louis

- th�se Yannick

- 2 posters ProVal

- pr�sentations, y compris celle-ci

- cours : Sylvain, JC
\end{frame}

\begin{frame}\frametitle{}
limitations / probl�mes / etc.

- limitations de \metapost\ : nombre de n\oe uds dans un chemin, etc.

- limitations de l'interpr�tation par pdflatex

- limitations de l'approche $\rightarrow$ calculs symboliques

une solution : reprogrammer l'�quivalent de \metapost

\bigskip

perte des fonctionnalit�s de r�solution d'�quations lin�aires de
\metapost

en partie remplac� par le placement relatif des bo�tes
\end{frame}



\end{document}

%%% Local Variables: 
%%% mode: latex
%%% ispell-local-dictionary: "francais"
%%% TeX-master: t
%%% End: 
