\documentclass[nodefaultblocks]{beamer}

\usepackage[latin1]{inputenc}
\usepackage[T1]{fontenc}
\usepackage[french]{babel}
\usepackage{mflogo,beamerthemesplit} 

\beamertemplatenavigationsymbolsempty
%\usecolortheme{crane}
\usetheme{Madrid}

\newcommand{\mlpost}{\textsc{Mlpost}}
\newcommand{\metapost}{\MP}
\newcommand{\nomdetikz}{\textsf{TikZ}}
\newcommand{\pstricks}{\textsf{PSTricks}}
\newcommand{\dia}{\textsf{Dia}}
\newcommand{\xfig}{\textsf{Xfig}}
\newcommand{\postscript}{PostScript}
\newcommand{\mlpictex}{mlP\hspace{-0.2em}\raisebox{-0.2em}{i}\hspace{-0.1em}c\hspace{-0.1em}\TeX}
\newcommand{\Item}{\bigskip\item}
\newcommand{\bred}[1]{\alert{\textbf{#1}}}
\let\emph\bred

\title[Faire bonne figure avec \mlpost]{\LARGE Faire bonne figure avec \mlpost}
\author[Bardou/Filli�tre/Kanig/Lescuyer]{R. Bardou
        \& J.-C. Filli�tre
        \& J. Kanig
        \& S. Lescuyer \\[1em]
      �quipe ProVal \\[0.5em]
      INRIA Saclay -- �le-de-France \\
      CNRS / LRI Universit� Paris-Sud }
%\institute[proval]{ProVal}
\date{JFLA  --- 3 f�vrier 2009}

\begin{document}

\begin{frame}
  \maketitle
  \begin{center}
    \includegraphics[width=0.3\textwidth]{ford.mps}
  \end{center}
\end{frame}

\begin{frame}\frametitle{Motivations 1/2}
comment r�aliser de belles figures contenant des �l�ments \LaTeX\ ?

\vfill

\begin{center}
  \includegraphics[width=\textwidth]{yannick1.mps}
\end{center}

\vfill

\begin{center}
  \includegraphics{another_list.mps}
\end{center}
\end{frame}

\begin{frame}\frametitle{Motivations 2/2}
comment r�aliser des figures utilisant des calculs ?

\vfill

\begin{columns}
\column{0.4\textwidth}  
\begin{center}
  \includegraphics{bresenham.mps}
\end{center}
\column{0.6\textwidth}  
\begin{center}
  \includegraphics[width=0.8\textwidth]{florence.mps}
\end{center}
\end{columns}
\end{frame}

\begin{frame}\frametitle{Solutions existantes}
  \begin{itemize}
  \item interfaces graphiques : \dia, \xfig
    \begin{itemize}
    \item[$+$] simple � utiliser (CQVVECQVO)
    \item[$-$] int�gration d'�l�ments \LaTeX\ difficile
    \end{itemize}
  \Item biblioth�ques \LaTeX\ : \pstricks, \nomdetikz/PGF
    \begin{itemize}
    \item[$+$] int�gration avec \LaTeX\ optimale
      % TODO S figure pstricks
    \item[$-$] programmation difficile
    \end{itemize}
  \Item outils externes : \metapost
    \begin{itemize}
    \item[$+$] tr�s bonne int�gration avec \LaTeX
    \item[$+$] langage sp�cifique
    \item[$-$] langage sp�cifique, avec de nombreux d�fauts
    \end{itemize}
  \end{itemize}
\end{frame}

\begin{frame}\frametitle{}
  notre solution
\end{frame}

\begin{frame}\frametitle{}
\begin{center}
\includegraphics[width=\textwidth]{stages.mps}
\end{center}
\end{frame}

\begin{frame}\frametitle{}
  d�mo
\end{frame}

\begin{frame}\frametitle{}

CHOIX DE CONCEPTION / API

- persistance

- arguments optionnels / �tiquet�s 

- variants polymorphes

- minimalit� de l'API

\end{frame}

\begin{frame}\frametitle{}
  \begin{center}
    \includegraphics{architecture.mps}
  \end{center}
\end{frame}

\begin{frame}\frametitle{}
  probl�mes techniques et solutions

  - d�pendances circulaires

  - limitations de \metapost\ $Rightarrow$ compilation

  - taille 
\end{frame}


\begin{frame}\frametitle{}
utilisations de \mlpost

- Florence, Louis

- th�se Yannick

- 2 posters ProVal

- pr�sentations, y compris celle-ci

- cours : Sylvain, JC
\end{frame}

\begin{frame}\frametitle{}
limitations / probl�mes / etc.

- limitations de \metapost\ : nombre de n\oe uds dans un chemin, etc.

- limitations de l'interpr�tation par pdflatex

- limitations de l'approche $\rightarrow$ calculs symboliques

une solution : reprogrammer l'�quivalent de \metapost

\bigskip

perte des fonctionnalit�s de r�solution d'�quations lin�aires de
\metapost

en partie remplac� par le placement relatif des bo�tes
\end{frame}



\end{document}

%%% Local Variables: 
%%% mode: latex
%%% ispell-local-dictionary: "francais"
%%% TeX-master: t
%%% End: 
