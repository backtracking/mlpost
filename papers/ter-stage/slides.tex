\documentclass{beamer}

\usepackage[utf8]{inputenc}
\usepackage[french]{babel}
\usepackage{url,graphicx,alltt}

\newcommand{\mlpost}{\textsc{Mlpost}}
\newcommand{\gmlpost}{\textsc{GMlpost}}
\newcommand{\meta}{\textsc{Metapost}}
\newcommand{\metapost}{\textsc{Metapost}}

\usetheme{Madrid}

\title{Extension de la bibliothèque Mlpost}
\author{J. Robert - G. Von Tokarski}
\date{15 mai 2009}

\begin{document}

\begin{frame}
  \maketitle

  \begin{center}
    TER Stage - Equipe \includegraphics[scale=0.4]{proval.png}
  \end{center}
\end{frame}

\begin{frame}{Problématique}
  Comment réaliser des figures
  \begin{itemize}
  \item nécessitant des calculs ?
  \begin{center}
    \includegraphics[scale=0.6]{ford.mps}
  \end{center}

  \bigskip
  \item incluant des éléments \LaTeX\ arbitraires ?
    \bigskip
    \begin{center}
      \includegraphics[scale=0.8]{graph.mps}
    \end{center}
  \end{itemize}
\end{frame}

\begin{frame}[fragile]\frametitle{\meta\ et \mlpost}
  \meta\ est un langage (de programmation) pour construire des dessins incluant des éléments \LaTeX

  \bigskip
  \mlpost\ est une interface Objective Caml de \meta

  \bigskip
  \begin{columns}
    \column{0.7\textwidth}\small
    \begin{verbatim}
  let simple_block =
    let b = Box.hblock ~pos:`Bot 
    [Box.tex "a"; Box.tex "A"; Box.tex "1"; 
     Box.tex "$\\pi$"] 
    in
    Box.draw b
    \end{verbatim}
    \column{0.3\textwidth}
    \includegraphics{simple_block.mps}
  \end{columns}
\end{frame}


\begin{frame}{Objectif du TER}
  Ajouter/Améliorer des modules haut niveau de \mlpost

  \vfill
  \begin{center}
    \includegraphics{architecture.mps}
  \end{center}

\end{frame}

% \begin{frame}
%   \frametitle{Nos contributions}
  
%   \begin{itemize}
%     \item Nouveaux modules
%       \begin{enumerate}
%       \item \texttt{Hist}
%       \item \texttt{Radar}
%       \end{enumerate}

%     \item Améliorations
%   \end{itemize}
% \end{frame}


\begin{frame}[fragile]{Histogramme Simple}
  \begin{columns}
    \column{0.5\linewidth}
    \begin{verbatim}
   let hist1 = 
     Hist.simple [3.;1.;6.]
    \end{verbatim}
    \column{0.5\linewidth}
    \begin{center}
      \includegraphics[scale=0.7]{hist1.mps}
    \end{center}
  \end{columns}
    
\end{frame}

\begin{frame}{Histogrammes comparatifs et cumulatifs}
  \begin{columns}
    \column{0.5\linewidth}
    \begin{center}
      \includegraphics% [scale=0.8]
      {hist2.mps}
    \end{center}
    
    \column{0.5\linewidth}
    \begin{center}
      \includegraphics% [scale=0.4]
      {hist3.mps}
    \end{center}
  \end{columns}
\end{frame}


\subsubsection{API}
\begin{frame}{API}
  \begin{center}
    \begin{alltt}
      Hist.f : arguments optionnels -> valeurs -> dessin
    \end{alltt}
  \end{center}

  \begin{columns}
    \column{0.5\linewidth}
    \begin{center}
      \begin{alltt}
        ?width:Num.t

        ?height:Num.t

        ?padding:Num.t

        ?fill:Color.t list

        ?perspective: bool

        ?hcaption:Picture.t

        ?vcaption:Picture.t

        ?histlabel:[> `Bot | `Center | `Top ] * Picture.t labels

        ?vlabel:Plot.labels

        ?hlabel:Picture.t list 
      \end{alltt}
    \end{center}
    
    \column{0.5\linewidth}
    \begin{center}
      \includegraphics[scale=0.7]{hist5.mps}
    \end{center}
  \end{columns}
\end{frame}
    
\begin{frame}{Structure du code}
  \begin{center}
    Code factorisé: les 3 types d'histogrammes sont principalement construits à partir de la même fonction.
  \end{center}
\end{frame}


\begin{frame}{Legend}
  \begin{center}
    \includegraphics% [scale=0.7]
    {legend1.mps}
  \end{center}
\end{frame}


\begin{frame}[fragile]{Radar cumulatif}
  \begin{columns}
    \column{0.5 \linewidth}
\begin{verbatim}
  val stack :
    ?radius:Num.t ->
    ?color:Color.t list ->
    ?pen:Pen.t ->
    ?style:Dash.t list ->
    ?ticks:float ->
    ?label:string list ->
    ?scale:float list ->
    float list list -> Picture.t
\end{verbatim}
    \column{0.5 \linewidth}\includegraphics [scale=0.6]{radar1.mps}
  \end{columns}
\end{frame}


\begin{frame}{Radar comparatif}
  \begin{center}
    \includegraphics [scale=0.6]
    {radar2.mps}
  \end{center}
\end{frame}

\begin{frame}[fragile]{Radar comparatif}
\begin{alltt}
  val compare :
    ?radius:Num.t ->
    ?color:Color.t list ->
    ?pen:Pen.t ->
    \color{red}?fill:bool ->
\color{black}    ?style:Dash.t list ->
    ?ticks:float ->
    ?label:string list ->
    ?scale:float list ->
    float list list -> Picture.t \color{red}list
\end{alltt}
\end{frame}


\begin{frame}[fragile]{Path}
  \begin{columns}
    \column{0.6 \linewidth}
\begin{verbatim}
type orientation = 
  | Up | Down | Left | Right
  | Upn of Num.t | Downn of Num.t 
  | Leftn of Num.t | Rightn of Num.t
  
val smart_path : ?style:joint -> orientation list 
  -> Point.t -> Point.t -> t
\end{verbatim}
\column{0.4 \linewidth}
\includegraphics {path4.mps}
\end{columns}
\end{frame}

\begin{frame}[fragile]{Path}
  \begin{columns}
    \column{0.7 \linewidth}
\begin{verbatim}
 let path = smart_path 
    ~style:jLine
    [Right;Down;Left;Down]
    (Point.pt (bp 0.,bp 0.)) 
    (Point.pt (bp 0.,bp (-30.)))
\end{verbatim}
    \column{0.3 \linewidth}
    \includegraphics [scale=1.2]
    {path2.mps}
\end{columns}
\end{frame}

\begin{frame}{Tree}
  \begin{center}
    \includegraphics [scale=1.2]
    {tree2.mps}
  \end{center}
\end{frame}

\begin{frame}[fragile]{API}
  \begin{columns}
    \column{0.7 \linewidth}
  \begin{verbatim}
 val leaf : Box.t -> t
 val node : ?ls:Num.t -> 
            ?cs:Num.t -> 
            ?pen:Pen.t -> 
            ?sep:Num.t ->
            ?stroke:Color.t -> 
            ?edge_style:edge_style -> 
            ?arrow_style:arrow_style -> 
            Box.t -> t list -> t
  \end{verbatim}
    \column{0.3 \linewidth}
    \includegraphics[scale=1.5]{tree5.mps}

    \bigskip
    \bigskip
    \includegraphics[scale=1.2]{tree4.mps}
    \end{columns}
\end{frame}

\begin{frame}{GMLPost}
\end{frame}

  


\end{document}

%%% Local Variables: 
%%% mode: latex
%%% mode: whizzytex
%%% mode: flyspell
%%% ispell-local-dictionary: "francais"
%%% End: 
