\documentclass{beamer}

\usepackage[utf8]{inputenc}
\usepackage[french]{babel}
\usepackage{url,graphicx,alltt}

\usetheme{Madrid}

\title{Extension de la bibliothèque Mlpost}
\author{J. Robert - G. Von Tokarski}
\date{15 mai 2009}

\begin{document}

\begin{frame}
  \maketitle
\end{frame}

\section*{Sommaire}
\begin{frame}
  \tableofcontents
\end{frame}

\section{MLPost}
\begin{frame}{Introduction à MLPost}
\end{frame}

\begin{frame}{Exemple de MLPost}
\end{frame}

\section{Notre TER}
\begin{frame}{Objectifs}
\end{frame}

\begin{frame}
  \frametitle{Nos modules}
  \begin{itemize}
    \item<1-> Hist
    \item<2-> Radar
    \item<3-> Path
    \item<4-> Tree
    \item<5-> Legend
  \end{itemize}
\end{frame}

\section{Nos modules}
\subsection{Hist}
\subsubsection{Simple}
\begin{frame}{Histogramme simple}
  
\end{frame}

\begin{frame}{Exemple}
  \begin{columns}
    \column{0.5\linewidth}
    \begin{alltt}
      let hist1 = Hist.simple
      \textasciitilde width:(bp 100.)
      \textasciitilde height:(bp 200.)
      [3.;1.;6.]
    \end{alltt}
    \column{0.5\linewidth}
    \begin{center}
      \includegraphics[scale=0.7]{hist1.mps}
    \end{center}
  \end{columns}
    
\end{frame}

\subsubsection{Comparatif}
\begin{frame}{Histogramme comparatif}
\end{frame}

\begin{frame}{Exemple}
  \begin{columns}
    \column{0.5\linewidth}
    \begin{alltt}
      let hist2 = 

      Hist.compare
      
      \textasciitilde width:(bp 100.)
      
      \textasciitilde height:(bp 200.)
      [[1.;5.;6.;5.;3.];
      [1.;2.;3.;6.;-1.]]
    \end{alltt}
    \column{0.5\linewidth}
    \begin{center}
      \includegraphics[scale=0.8]{hist2.mps}
    \end{center}
  \end{columns}
\end{frame}

\subsubsection{Cumulatif}
\begin{frame}{Histogramme cumulatif}
\end{frame}

\begin{frame}{Exemple}
  \begin{columns}
    \column{0.5\linewidth}
    \begin{alltt}
      let hist3 =

      let vlabel \_ \_ = None in

      Hist.stack
 
      \textasciitilde vlabel

      \textasciitilde fill:[lightred;lightblue;
      lightyellow;lightgreen]

      [[4.;5.;5.;]; [8.;3.;1.]; [2.;8.;1.;4.];
      [1.5;3.5];[2.;2.;7.;1.]]
    \end{alltt}
    \column{0.5\linewidth}
    \begin{center}
      \includegraphics[scale=0.4]{hist3.mps}
    \end{center}
  \end{columns}
\end{frame}

\subsubsection{Exemple complexe}
\begin{frame}{Exemple complexe}
\end{frame}

\subsection{Radar}
\subsubsection{Cumulatif}
\begin{frame}{Radar cumulatif}
\end{frame}

\begin{frame}{Exemple}
\end{frame}

\subsubsection{Comparatif}
\begin{frame}{Radar comparatif}
\end{frame}

\begin{frame}{Exemple}
\end{frame}

\subsection{Path}
\begin{frame}{Path}
\end{frame}

\begin{frame}{Exemple}
\end{frame}

\subsection{Tree}
\begin{frame}{Tree}
\end{frame}

\begin{frame}{Exemple}
\end{frame}

\subsection{Legend}
\begin{frame}{Legend}
\end{frame}

\begin{frame}{Exemple}
\end{frame}

\section{GMLPost}
\begin{frame}{GMLPost}
\end{frame}

  % \begin{columns}
%     \column{0.5\textwidth}
%     qdqd kjhqd jkqd kjhqdkj

%     \begin{center}
%       \includegraphics{simple_block.mps}
%     \end{center}
%     hqd lk kljsqdlkqdlk

%     \column{0.5\textwidth}
%     deuxième colonne

%     \begin{alltt}
%       let f x = x+1

%       let g x = x
%     \end{alltt}
%   \end{columns}


\end{document}

%%% Local Variables: 
%%% mode: latex
%%% mode: whizzytex
%%% mode: flyspell
%%% ispell-local-dictionary: "francais"
%%% End: 
