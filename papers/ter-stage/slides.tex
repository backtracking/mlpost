\documentclass{beamer}

\usepackage[utf8]{inputenc}
\usepackage[french]{babel}
\usepackage{url,graphicx,alltt}

\newcommand{\mlpost}{\textsc{Mlpost}}
\newcommand{\gmlpost}{\textsc{GMlpost}}
\newcommand{\meta}{\textsc{Metapost}}
\newcommand{\metapost}{\textsc{Metapost}}

\usetheme{Madrid}

\title{Extension de la bibliothèque Mlpost}
\author{J. Robert - G. Von Tokarski}
\date{15 mai 2009}

\begin{document}

\begin{frame}
  \maketitle

  \begin{center}
    TER Stage - Equipe \includegraphics[scale=0.4]{proval.png}
  \end{center}
\end{frame}

\begin{frame}{Problématique}
  Pour l'utilisateur de \LaTeX, comment réaliser des figures
  \begin{itemize}
  \item nécessitant des calculs ?
  \begin{center}
    \includegraphics[scale=0.6]{ford.mps}
  \end{center}

  \bigskip
  \item incluant des éléments \LaTeX\ arbitraires ?
    \bigskip
    \begin{center}
      \includegraphics[scale=0.8]{graph.mps}
    \end{center}
  \end{itemize}
\end{frame}

\begin{frame}[fragile]\frametitle{\meta\ et \mlpost}
  \meta\ est un langage (de programmation) pour construire des dessins incluant des éléments \LaTeX

  \bigskip
  \mlpost\ est une interface Objective Caml de \meta

  \bigskip
  \begin{columns}
    \column{0.7\textwidth}\small
    \begin{verbatim}
  let simple_block =
    let b = Box.hblock ~pos:`Bot 
    [Box.tex "a"; Box.tex "A"; Box.tex "1"; 
     Box.tex "$\\pi$"] 
    in
    Box.draw b
    \end{verbatim}
    \column{0.3\textwidth}
    \includegraphics{simple_block.mps}
  \end{columns}
\end{frame}


\begin{frame}{Objectif du TER}
  Ajouter/Améliorer des modules haut niveau de \mlpost

  \vfill
  \begin{center}
    \includegraphics{architecture.mps}
  \end{center}

\end{frame}


\begin{frame}[fragile]{Histogramme Simple}
  \begin{columns}
    \column{0.5\linewidth}
    \begin{verbatim}
   let hist = 
     Hist.simple [3.;1.;6.]
    \end{verbatim}
    \column{0.5\linewidth}
    \begin{center}
      \includegraphics[scale=0.7]{hist1.mps}
    \end{center}
  \end{columns}
    
\end{frame}

\begin{frame}{Histogrammes comparatifs et cumulatifs}
  \begin{columns}
    \column{0.5\linewidth}
    \begin{center}
      \includegraphics% [scale=0.8]
      {hist2.mps}
    \end{center}
    
    \column{0.5\linewidth}
    \begin{center}
      \includegraphics% [scale=0.4]
      {hist3.mps}
    \end{center}
  \end{columns}
\end{frame}


\begin{frame}{API}
  \begin{center}
    \begin{alltt}
      \color{red}hist :
      \color{blue}
      ?width:Num.t ->

      ?height:Num.t ->

      ?padding:Num.t ->

      ?fill:Color.t list ->

      ?perspective: bool ->

      ?hcaption:Picture.t ->

      ?vcaption:Picture.t ->

      ?histlabel:[> `Bot | `Center | `Top ] 
      
      * Picture.t list labels ->

      ?vlabel:Plot.labels ->
      
      ?hlabel:Picture.t list -> 
      
      \color{red}données -> dessin
    \end{alltt}
    \end{center}
\end{frame}

\begin{frame}[fragile]{Exemple d'utilisation}
\begin{columns}
  \column{0.5\linewidth}
  \begin{center}\footnotesize{
\begin{verbatim}
  let hist =
    let vlabel _ _ = None in
    let rot s = Picture.rotate 25. 
      (Picture.tex s) in
    Hist.stack
    ~vlabel
    ~perspective:true 
    ~padding:(bp 15.)
    ~fill:[lightred;lightblue;
      lightyellow;lightgreen]
    ~histlabel:(`Center, Hist.Values)
    ~vcaption:(Picture.tex "Dollars")
    ~hlabel:[rot "2007";rot "2008";rot "2009"]
    [[4.;5.;5.;]; [8.;3.;1.]; [2.;8.;1.;4.]]
\end{verbatim}}
      \end{center}
    
    \column{0.5\linewidth}
    \begin{center}
      \includegraphics[scale=0.7]{hist5.mps}
    \end{center}
  \end{columns}
\end{frame}
    
\begin{frame}{Structure du code}
  \begin{itemize}
  \item Code factorisé: les 3 types d'histogrammes sont principalement construits à partir de la même fonction.
  
  \bigskip
  \item Cette fonction générale prend en paramètre une liste de liste et construit un histogramme cumulatif.
  
  \bigskip
  \item Les histogrammes simples et comparatifs sont des cas particuliers.
  \end{itemize}
\end{frame}


\begin{frame}{Legend}
  \begin{columns}
    \column{0.8 \linewidth}
    \begin{center}
      \includegraphics {hist3.mps}
    \end{center}
    \column{0.2 \linewidth} \includegraphics [scale=0.8]{legend1.mps}
  \end{columns}
\end{frame}

\begin{frame}{Radar comparatif}
  \begin{center}
    \includegraphics [scale=0.6]
    {radar2.mps}
  \end{center}
\end{frame}

\begin{frame}[fragile]{Radar cumulatif}
  \begin{columns}
    \column{0.5 \linewidth}
\begin{verbatim}
  val stack :
    ?radius:Num.t ->
    ?color:Color.t list ->
    ?pen:Pen.t ->
    ?style:Dash.t list ->
    ?ticks:float ->
    ?label:string list ->
    ?scale:float list ->
    float list list -> Picture.t
\end{verbatim}
    \column{0.5 \linewidth}\includegraphics [scale=0.6]{radar1.mps}
  \end{columns}
\end{frame}


\begin{frame}[fragile]{Radar comparatif}
\begin{alltt}
  val compare :
    ?radius:Num.t ->
    ?color:Color.t list ->
    ?pen:Pen.t ->
    \color{red}?fill:bool ->
\color{black}    ?style:Dash.t list ->
    ?ticks:float ->
    ?label:string list ->
    ?scale:float list ->
    float list list -> Picture.t \color{red}list
\end{alltt}
\end{frame}


\begin{frame}[fragile]{Path}
  \begin{columns}
   % \column{0.4 \linewidth}
    
    \bigskip
    \bigskip
    \column{0.8 \linewidth}
    \begin{center}
    \includegraphics {path5.mps}
    \bigskip
    
    \includegraphics {path4.mps}
  \end{center}
    \column{0.2 \linewidth}
    \includegraphics {path1.mps}
  \end{columns}
\end{frame}


\begin{frame}[fragile]{API}
\begin{verbatim}
type orientation = 
  | Up | Down | Left | Right
  | Upn of Num.t | Downn of Num.t 
  | Leftn of Num.t | Rightn of Num.t
  
val smart_path : ?style:joint -> orientation list 
  -> Point.t -> Point.t -> t
\end{verbatim}
\end{frame}


\begin{frame}[fragile]{Path}
  \begin{columns}
    \column{0.7 \linewidth}
\begin{verbatim}
 let path = smart_path 
    ~style:jLine
    [Right;Down;Left;Down]
    (Point.pt (bp 0.,bp 0.)) 
    (Point.pt (bp 0.,bp (-30.)))
\end{verbatim}
    \column{0.3 \linewidth}
    \includegraphics [scale=1.2]
    {path2.mps}
\end{columns}
\end{frame}

\begin{frame}{Tree}
  \begin{columns}
    \column{0.5\textwidth}
    \begin{center}
      \includegraphics [scale=1.2]
      {tree2simple.mps}
    \end{center}
    \column{0.5\textwidth}
    \begin{center}
      \includegraphics [scale=1.2]
      {tree2.mps}
    \end{center}
  \end{columns}
\end{frame}

\begin{frame}[fragile]{API}
  \begin{columns}
    \column{0.7 \linewidth}
  \begin{verbatim}
 val leaf : Box.t -> t
 val node : ?ls:Num.t -> 
            ?cs:Num.t -> 
            ?pen:Pen.t -> 
            ?sep:Num.t ->
            ?stroke:Color.t -> 
            ?edge_style:edge_style -> 
            ?arrow_style:arrow_style -> 
            Box.t -> t list -> t
  \end{verbatim}
    \column{0.3 \linewidth}
    \includegraphics[scale=1.5]{tree5.mps}

    \bigskip
    \bigskip
    \includegraphics[scale=1.2]{tree4.mps}
    \end{columns}
\end{frame}

\begin{frame}{GMlpost - Motivations}
  \begin{itemize}
  \item<1-> \mlpost\ permet de fabriquer des figures jolies et personnalisées.
  
  \bigskip
  
  Or il est souvent nécessaire de recompiler plusieurs fois la figure en changeant certaines valeurs pour se rapprocher du rendu souhaité. 

\bigskip

  \item<2->  \gmlpost\ est une interface graphique de \mlpost.

    \bigskip

    \gmlpost\ permet d'éditer interactivement des valeurs et des points choisis par l'utilisateur.

  \end{itemize}
\end{frame}

\begin{frame}{Screenshot}
\includegraphics[scale=0.4]{screen2.png}
\end{frame}


\begin{frame}{Solution technique}
  \begin{columns}
    \column{0.5 \linewidth}
    \begin{center}
      \includegraphics [scale=0.7]{interface1.mps}
    \end{center}
    \column{0.5 \linewidth}
    \begin{center}
      \includegraphics[scale=0.7]{interface2.mps}
    \end{center}
  \end{columns}
\end{frame}

\begin{frame}{Conclusion}
  \begin{itemize}
  \item contribution significative à \mlpost
    \begin{itemize}
    \item<1-> utilisée dès à présent dans l'équipe ProVal
    \item<2-> sera distribuée avec la prochaine version de \mlpost
    \end{itemize}

    \bigskip
  \item sur un plan personnel
    \begin{itemize}
      \item<3-> utilisation différente d'Objective Caml
      \item<4-> découverte de lablgkt2
    \end{itemize}
  \end{itemize}
\end{frame}

\end{document}

%%% Local Variables: 
%%% mode: latex
%%% mode: whizzytex
%%% mode: flyspell
%%% ispell-local-dictionary: "francais"
%%% End: 
