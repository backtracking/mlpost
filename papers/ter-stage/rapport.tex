\documentclass[a4paper,12pt]{article}

\usepackage[utf8]{inputenc}
\usepackage[french]{babel}
\usepackage{fullpage,url,graphicx,alltt}

\newcommand{\mlpost}{\textsc{Mlpost}}

\begin{document}

\section{Introduction}

\subsection{Présentation de \mlpost}
\mlpost\ est une interface Objective Caml de Metapost, permettant la création de figures.
\subsubsection{Motivations}
Dans un cadre scientifique, il est très souvent nécessaire d'inclure des figures aux documents (cours, articles).
~\\Une façon très commune de rédiger ces documents est d'utiliser \LaTeX. \LaTeX\ permet l'importation d'images, voire la création, mais ceci reste compliqué, long, et les erreurs sont difficiles à interpréter. Il doit également être possible d'ajouter des éléments de \LaTeX\ à la figure, et de garder une homogénéité du document. 

~\\L'utilisation d'un langage comme OCaml, avec un typage fort et des erreur plus simples à déchiffrer s'imposait donc! De plus, OCaml bénéficie d'arguments optionnels, très pratiques dans la conception d'une librairie graphique.Enfin, la quantité de code tapé devient plus concise, rendant la conception de figures possible d'un simple appel de fonction.

\subsubsection{Exemples}

\includegraphics{simple_block.mps}



\subsection{Notre TER}
\subsubsection{Modules au dessus de \mlpost}
\subsubsection{Interface Graphique}

\section{Modules \mlpost}


\subsection{Histogrammes}
Il existe trois types d'histogramme:

\subsubsection{Simple:}
Un histogramme classique construit à partir d'une liste de nombres
flottants.

~\\

\begin{minipage}{0.5\linewidth}
  \begin{alltt}
    let hist = Hist.simple [3.;1.;6.]
  \end{alltt}
\end{minipage}
\begin{minipage}{0.5\linewidth}
\begin{center}
\includegraphics[scale=0.5]{hist1.mps}
\end{center}
\end{minipage}

\subsubsection{Comparatif :} 
Un histogramme comparatif construit à partir d'une liste. Cette liste
contient N listes de nombres flottants, chacune étant un histogramme.

~\\

\begin{minipage}{0.5\linewidth}
  \begin{alltt}
    let hist = Hist.compare
    [[1.;5.;6.;5.;3.];
      [1.;2.;3.;6.;-1.]]
  \end{alltt}
\end{minipage}
\begin{minipage}{0.5\linewidth}
\begin{center}
\includegraphics[scale=0.5]{hist2.mps}
\end{center}
\end{minipage}

\subsubsection{Cumulatif :} 
Un histogramme cumulatif.

~\\

\begin{minipage}{0.5\linewidth}
  \begin{alltt}
    let hist = Hist.stack 
    ~fill:[lightred;lightblue
      ;lightyellow;lightgreen]
    [[4.;5.;5.;]; [8.;3.;1.]; [2.;8.;1.;4.]]
  \end{alltt}
\end{minipage}
\begin{minipage}{0.5\linewidth}
\begin{center}
\includegraphics[scale=0.5]{hist3.mps}
\end{center}
\end{minipage}

\subsubsection{Pour aller plus loin:}
Il est possible de faire des histogrammes bien plus complexes, mais toujours de façon très simple.

~\\

\begin{minipage}{0.5\linewidth}
  \begin{alltt}
    let hist =
    let pics =
    List.map Picture.tex ["2000";"2001";
      "2002";"2003";"2004";"2005"]
    in
    Hist.simple 
    ~histlabel:(`Top, Hist.User pics)
    [4.5;5.0;6.2;8.;7.2;6.1]
  \end{alltt}
\end{minipage}
\begin{minipage}{0.5\linewidth}
\begin{center}
\includegraphics[scale=0.5]{hist4.mps}
\end{center}
\end{minipage}

~\\

\begin{minipage}{0.5\linewidth}
  \begin{alltt}
    let hist = Hist.stack
    ~perspective:true ~padding:(bp 25.)
    ~fill:[lightred;lightblue;
      lightyellow;lightgreen]
    ~histlabel:(`Center, Hist.Values)
    [[4.;5.;5.;]; [8.;3.;1.]; [2.;8.;1.;4.]]
  \end{alltt}
\end{minipage}
\begin{minipage}{0.5\linewidth}
\begin{center}
\includegraphics[scale=0.5]{hist5.mps}
\end{center}
\end{minipage}

\subsection{Radar}
Ce module permet la création de diagrammes radar.
Nous donnons la possibilité d'en créer deux sortes.
Lorsqu'un radar est créé, l'objet \mlpost\ renvoyé est une Picture.
Nous en verrons l'utilité plus tard.

\subsubsection{Cumulatif:}
Le diagramme radar classique. Toutes les données sont regroupées sur la même figure, de façon à les comparer instantanément. 

~\\

\begin{minipage}{0.5\linewidth}
  \begin{alltt}
    let radar =
    let pic =
    Radar.stack
    ~pen:(Pen.scale (bp 3.) Pen.circle)
    ~color:[blue;red;green]
    ~label:["weight";"acceleration";
      "speed";"maniability";"stickiness"]
    [[3.;4.;5.;6.;4.];
      [6.;5.;2.;1.;1.];
      [1.;7.;2.;4.;5.]]
    in
    Command.draw_pic pic
  \end{alltt}
\end{minipage}
\begin{minipage}{0.5\linewidth}
\begin{center}
\includegraphics[scale=0.8]{radar1.mps}
\end{center}
\end{minipage}

\subsubsection{Comparatif}
Contrairement au diagramme précédent, celui ci ne fabrique pas qu'une seule Picture, mais autant qu'il existe d'objets à comparer. Cela permet donc un plus grand choix dans la mise en page.

~\\

\begin{minipage}{0.5\linewidth}
  \begin{alltt}
    let radar =
    let pics =
    Radar.compare
    ~pen:(Pen.scale (bp 1.5) Pen.circle)
    ~color:[lightblue;lightred;lightgreen] 
    ~fill:true
    [[3.;4.;5.;6.;4.];
      [6.;5.;2.;1.;1.];
      [1.;7.;2.;4.;5.]]
    in
    Box.draw (Box.vbox ~padding:(bp 10.) 
    (List.map (Box.pic ~stroke:None) pics))

  \end{alltt}
\end{minipage}
\begin{minipage}{0.5\linewidth}
\begin{center}
\includegraphics[scale=0.5]{radar2.mps}
\end{center}
\end{minipage}

\subsection{Path}
Dans \mlpost, le module Path permet à l'utilisateur de tracer toutes sortes de lignes, qu'elles soient droites ou des courbes de Bézier, et ce par le biais de nombreuses fonctions.
Nous n'avons rien modifié de ce qui était déjà en place. Nous avons seulement ajouter une nouvelle manière de tracer un Path, appelée \textit{smart\_path}. 

~\\Cette fonction prend en paramètre deux points: le point de départ du Path, et le point d'arrivée. 
\begin{alltt}
  val smart_path : ?style:joint -> orientation list -> Point.t -> Point.t -> t
\end{alltt}
~\\La particularité de \textit{smart\_path} est que la fonction prend un troisième paramètre, qui est une liste d'orientations.
~\\Le type orientation se définit comme ceci:
\begin{alltt}
  type orientation =
  Up | Down | Left | Right |
  Upn of Num.t | Downn of Num.t | Leftn of Num.t | Rightn of Num.t
\end{alltt}
Ceci permet à l'utilisateur de choisir la façon dont il veut relier les deux points. Dans la mesure du possible, \mlpost\ créera ce path, ou retournera une erreur. 

~\\exemples de \textit{smart\_path}:

~\\
\begin{minipage}{0.5\linewidth}
  \begin{alltt}
    let path = 
    let p = Path.smart_path 
    ~style:jLine
    [Right;Down;Right]
    (Point.pt (bp 10.,bp 20.)) 
    (Point.pt (bp 100.,bp 100.))
    in
    Command.draw p
  \end{alltt}
\end{minipage}
\begin{minipage}{0.5\linewidth}
\begin{center}
\includegraphics[scale=0.5]{path1.mps}
\end{center}
\end{minipage}

\subsection{Arbres}

\cite{tree}

\section{Interface graphique}
\subsection{Motivations}
\subsection{Structure}
\subsection{Screenshots}
\section{Conclusion}

\begin{thebibliography}{99}
\bibitem{mlpost} \mlpost, une bibliothèque de dessin 
 scientifique pour Objective Caml.

\bibitem{tree}
Andrew Kennedy. 
\emph{Drawing Trees.}
Journal of Functional Programming, 
6(3): 527--534, Cambridge University Press, May 1996.
\end{thebibliography}

\end{document}

%%% Local Variables: 
%%% mode: latex
%%% mode: whizzytex
%%% mode: flyspell
%%% ispell-local-dictionary: "francais-latin1"
%%% End: 
