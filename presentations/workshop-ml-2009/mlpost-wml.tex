\documentclass{article}
\usepackage[utf8]{inputenc}
\usepackage[T1]{fontenc}
\usepackage{graphicx}
\usepackage{url}

\newcommand{\mlpost}{Mlpost}
\newcommand{\ocaml}{Ocaml}
\newcommand{\metapost}{Metapost}
\newcommand{\tikz}{TikZ}
\title{Mlpost - A scientific drawing library}
\date{}
\author{Romain Bardou ~\hfill~ François Bobot \\ Jean-Christophe
  Filliâtre ~\hfill~ Johannes Kanig \\ Stéphane Lescuyer}
\begin{document}
\maketitle

We present \mlpost, an \ocaml\ library for drawing diagrams. Because
of the possibility to include arbitrary \LaTeX\ text in figures
and the resulting figure in \LaTeX\ documents, we believe
it is particularly suited for the scientific community and ML
programmers. Unlike other tools with similar features and scope,
like \metapost\ or \tikz, \mlpost\ is {\em not} a stand-alone
language, but a library for the modern \ocaml\ programming language.
This greatly eases the construction of figures resulting from a
computation or figures which are intended to illustrate the run of an
algorithm.

Initially based on the \metapost\ tool to do the actual
computations, treatment of \LaTeX\ snippets and drawing, \mlpost\ has
grown into a full-blown implementation of Bézier curves and interprets
results of \LaTeX\ runs by itself, using the {\tt Cairo} library
to draw objects. It supports output in PDF, Postscript, SVG,
and even to X11 windows. Mlpost is freely distributed at \url{http://mlpost.lri.fr}.

%TODO decide what we want to show
The demo, if accepted, will show many examples from lectures, PhD
theses, articles, talks, etc. In particular, it will focus on figures
which illustrate the run of an algorithm. It will also show the basic
principles of the \mlpost\ library.


%TODO references
\subsection*{Contact Information}
%TODO more contact information? someone else?

Johannes Kanig\\
INRIA Saclay Île de France - ProVal\\
Parc Club Orsay Université, bâtiment N\\
4 rue Jacques Monod\\
F-91893 Orsay cedex France\\
Tel.: (+33) (0)1 72 92 59 79\\
johannes.kanig@lri.fr



\end{document}
