\documentclass[nodefaultblocks]{beamer}
\usepackage[utf8]{inputenc}
\usepackage[T1]{fontenc}
\usepackage{beamerthemesplit}
\usepackage{stmaryrd}
\usepackage{amsfonts}
\usepackage{latexsym}
\usepackage{amssymb}
\usepackage{amsmath}
\usepackage{mathrsfs}
\usepackage{pifont}
\usepackage{shadow}   
\usepackage{alltt}   
\usepackage[dvips]{epsfig}

\usetheme{Madrid}

\beamertemplatenavigationsymbolsempty

\definecolor{yel}{rgb}{.9,.8,.4}
\definecolor{mygray}{rgb}{0.85,0.85,0.85}

\def \ebox #1{\colorbox{yel}{\mbox {\strut #1}}}
\def \gbox #1{\colorbox{mygray}{\mbox {\strut #1}}}

\title{MlPost}
\author{The MlPost team}
\institute{Proval}
\date{Réunion Proval}

\begin{document}

\begin{frame}
  
  \begin{center}
    {\huge{MlPost - un outil de dessin scientifique}}\\[2em]
    {\large \alert{Jean-Christophe Filli\^atre, Johannes Kanig, St\'ephane Lescuyer}}
    \medskip
    \hfill\includegraphics[height=5cm]{cfg.mps}
  \end{center}
  \hfill\includegraphics{proval.mps}
%  \hfill\includegraphics[width=0.15\textwidth]{lrilogo.jpg}ρ
\end{frame}

\begin{frame}\frametitle{Motivation}

  \begin{block}{Problème}
    Un outil capable de combiner
    \begin{itemize}
      \item des éléments de dessin souvent simples
      \item du Latex, et prise en compte de sa taille
      \item Automatisation commode (dessin de grille, placement d'arbre)
    \end{itemize}
  \end{block}

  \begin{block}{Solutions existantes}
    \begin{itemize}
      \item outils graphiques gratuits style {\tt dia} : Pas de support Latex,
        pas d'automatisation
      \item outils ``langages de programmation'': Metapost, Tikz
    \end{itemize}
  \end{block}
\end{frame}

\begin{frame}\frametitle{Metapost}
  \begin{minipage}{0.3\linewidth}
  \includegraphics{mpex.mps}
  \end{minipage}
  \hfill
  \begin{minipage}{0.6\linewidth}
    \small
    \begin{alltt}
beginfig(1);\\
\hspace*{0pt}~ numeric u;\\
\hspace*{0pt}~ u = 1cm;\\
\hspace*{0pt}~ draw (0,2u)---(0,0)---(4u,0);\\
\hspace*{0pt}~ pickup pencircle scaled 1pt;\\
\hspace*{0pt}~ draw (0,0)\{up\}\\
\hspace*{0pt}~ for i=1 upto 8: \\
\hspace*{0pt}~~~  ..(i/2,sqrt(i/2))*u  \\
\hspace*{0pt}~ endfor;\\
\hspace*{0pt}~ label.lrt(btex $\sqrt x$ etex, 
\hspace*{0pt}~ ~ ~ ~ ~ ~ (3,sqrt 3)*u);\\
\hspace*{0pt}~ label.bot(btex $x$ etex, (2u,0));\\
\hspace*{0pt}~ label.lft(btex $y$ etex, (0,u));\\
endfig;\\
    \end{alltt}
  \end{minipage}
\end{frame}

\begin{frame}\frametitle{Tikz}
%\begin{tikzpicture}[scale=2]
%  \shade[top color=blue,bottom color=gray!50] 
%      (0,0) parabola (1.5,2.25) |- (0,0);
%  \draw (1.05cm,2pt) node[above] 
%      {$\displaystyle\int_0^{3/2} \!\!x^2\mathrm{d}x$};
%
%  \draw[style=help lines] (0,0) grid (3.9,3.9)
%       [step=0.25cm]      (1,2) grid +(1,1);
%
%  \draw[->] (-0.2,0) -- (4,0) node[right] {$x$};
%  \draw[->] (0,-0.2) -- (0,4) node[above] {$f(x)$};
%
%  \foreach \x/\xtext in {1/1, 1.5/1\frac{1}{2}, 2/2, 3/3}
%    \draw[shift={(\x,0)}] (0pt,2pt) -- (0pt,-2pt) node[below] {$\xtext$};
%
%  \foreach \y/\ytext in {1/1, 2/2, 2.25/2\frac{1}{4}, 3/3}
%    \draw[shift={(0,\y)}] (2pt,0pt) -- (-2pt,0pt) node[left] {$\ytext$};
%
%  \draw (-.5,.25) parabola bend (0,0) (2,4) node[below right] {$x^2$};
%\end{tikzpicture}
\end{frame}

\begin{frame}\frametitle{Pourquoi pas MetaPost ou Tikz ?}
  \begin{block}{MetaPost}
    \begin{itemize}
      \item Pas de structures de données
      \item Pas de fonctions (seulement macros textuelles) 
      \item précédences des opérateurs très bizarres
    \end{itemize}
  \end{block}
  \begin{block}{Tikz}
    \begin{itemize}
      \item langage très complet, similaire dans l'esprit à MetaPost
      \item les dessins se font directement en \LaTeX
      \item compliqué à prendre en main 
    \end{itemize}
  \end{block}

  \medskip

  les deux : Encore un langage à apprendre, messages d'erreur
  incompréhensibles, \dots
\end{frame}

\begin{frame}\frametitle{MlPost}
  \begin{block}{A commencé comme un ``pretty-printeur'' de Metapost}
    \begin{itemize}
      \item pour profiter des capacités de Metapost 
      \item typage et expressivité de Caml
      \item developpement facilité d'extensions
    \end{itemize}
  \end{block}

  \begin{block}{Aujourd'hui : bibliothèque OCaml}
    \begin{itemize}
      \item un ``cross-compiler'' vers Metapost
      \item interface fonctionnelle
      \item memoization des différents motifs d'un dessin
      \item utiliser le côté impératif de MetaPost pour du partage
    \end{itemize}
  \end{block}

\end{frame}

\begin{frame}\frametitle{Brique de base : les chemins}
  \begin{block}{Les chemins}
    \begin{itemize}
      \item Une liste de points, connectés par des lignes droites, des courbes
        de Bézier, \dots
      \item on peut les fermer, les combiner, les intersecter, remplir la
        surface délimitée, s'en servir pour faire du {\em clipping}, \dots
    \end{itemize}
  \end{block}
  \begin{center}
  \includegraphics{path.mps}
  \end{center}
\end{frame}

\begin{frame}\frametitle{Types de base}

  \begin{block}{Types différents}
    \begin{itemize}
      \item longueurs : {\tt Num}
      \item points : {\tt Point}
      \item chemins : {\tt Path}
      \item transformations affines : {\tt Transform}
      \item plumes : {\tt Pen}
      \item couleurs : {\tt Color}
    \end{itemize}
  \end{block}

  \begin{center}
    \includegraphics[width=0.3\textheight]{colortriangle.mps}
  \end{center}
  
\end{frame}

\begin{frame}
  boites, flêches, labels, latex \dots
\end{frame}

\begin{frame}\frametitle{Pictures}

\end{frame}

\begin{frame}
  extensions: Diag, Tree, Plot
\end{frame}

\begin{frame}\frametitle{Conclusion}

  \begin{block}{Points forts de MlPost}
    \begin{itemize}
      \item les mêmes que MetaPost ! 
      \item Bibliothèque Caml - dans votre langage de programmation favori
      \item The sky is the limit : on attend vos extensions
      \item Supporté par JCF  :)
    \end{itemize}
  \end{block}
  
  \begin{block}{Inconvénients}
    \begin{itemize}
      \item Syntaxe un peu lourde (inévitable, sans réinventer un langage),
        camlp4~?
      \item Pas toutes les fonctionnalités de MetaPost, peu d'extensions
        comparé à Tikz
      \item Toutes les erreurs ne peuvent pas être détectées par typage
      \item Pas encore prouvé en Coq :)
    \end{itemize}
  \end{block}
\end{frame}

\end{document}
